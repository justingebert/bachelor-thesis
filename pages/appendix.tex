\appendix

\chapter{Appendix} \label{chapter:appendix}

\section{Source Code and Data}
The source code of the thesis is available on GitHub split in two repositories. Both repositories have Readme files that explain how to use the code and data.
\begin{itemize}
    \item \textbf{Prototype Repository:} Contains the implementation of the APR Core and the CI configuration. It is available at: \url{https://github.com/justingebert/bugfix-ci}
    \item \textbf{Evaluation Repository:} Contains the evaluation scripts and results of the APR system, including the QuixBugs benchmark setup. It is available at: \url{https://github.com/justingebert/quixbugs-apr}
\end{itemize}


\section{LLM Versions}
The following Table lists the exact LLM versions that where used during development and evaluation.
\begin{longtable}{p{5cm} | p{6cm}}
    \caption{LLM models and their versions used in the system} \label{table:llm_versions} \\
    \hline
    \textbf{Model Name}   & \textbf{Version}                                              \\
    \hline
    \endfirsthead
    \hline
    \endfoot
    gemini-2.0-flash-lite & gemini-2.0-flash-lite                                         \\
    gemini-2.0-flash      & gemini-2.0-flash                                              \\
    gemini-2.5-flash-lite & gemini-2.5-flash-lite-preview-06-17                           \\
    gemini-2.5-flash      & gemini-2.5-flash                                              \\
    gemini-2.5-pro        & gemini-2.5-pro                                                \\
    gpt-4.1-nano          & gpt-4.1-nano-2025-04-14                                       \\
    gpt-4.1-mini          & gpt-4.1-mini-2025-04-14                                       \\
    gpt-4.1               & gpt-4.1-2025-04-14                                            \\
    o4-mini               & o4-mini-2025-04-16                                            \\
    claude-3-5-haiku      & claude-3-5-haiku-20241022                                     \\
    claude-3-7-sonnet     & claude-3-7-sonnet-20250219                                    \\
    claude-sonnet-4-0     & claude-sonnet-4-20250514                                      \\
    \hline
\end{longtable}

\newpage

\includepdf[pages={1}]{pages/Ai-usage-card.pdf}

\thispagestyle{empty}
%\vspace*{18cm}
\noindent


\section*{Eidesstattliche Versicherung}
Hiermit versichere ich an Eides statt durch meine Unterschrift, dass ich die vorstehende Arbeit selbstst\"andig und ohne fremde Hilfe angefertigt und alle Stellen, die ich w\"ortlich oder ann\"ahernd w\"ortlich aus Ver\"offentlichungen entnommen habe, als solche kenntlich gemacht habe, mich auch keiner anderen als der angegebenen Literatur oder sonstiger Hilfsmittel bedient habe. Die Arbeit hat in dieser oder \"ahnlicher Form noch keiner anderen Pr\"ufungsbeh\"orde vorgelegen.\\

\textbf{Angaben zur Verwendung KI-basierter Hilfsmittel}

Im Anhang meiner Arbeit habe ich s\"amtliche KI-basierte Hilfsmittel angegeben. Diese sind mit Produktnamen und Anwendung in einem KI-Verzeichnis ausgewiesen. Ich versichere, dass ich keine KI-basierten Tools verwendet habe, deren Nutzung der Pr\"ufer / die Pr\"uferin explizit schriftlich ausgeschlossen hat. Ich bin mir bewusst, dass die Verwendung von Texten oder anderen Inhalten und Produkten, die durch KI-basierte Tools generiert wurden, keine Garantie f\"ur deren Qualit\"at darstellt. Ich verantworte die \"Ubernahme jeglicher von mir verwendeter maschinell generierter Passagen vollumf\"anglich selbst und trage die Verantwortung f\"ur eventuell durch die KI generierte fehlerhafte oder verzerrte Inhalte, fehlerhafte Referenzen, Verst\"osse gegen das Datenschutz- und Urheberrecht oder Plagiate. Ich versichere zudem, dass in der vorliegenden Arbeit mein gestalterischer Einfluss \"uberwiegt. \\
\linebreak[4]
\linebreak[4]
\linebreak[4]
\linebreak[4]
-------------------------------------------------------\linebreak[4]
Datum, Ort, Unterschrift
\thispagestyle{empty}       % keine Seitennummer

\section*{Abstract}
Generative AI technologies are reshaping software engineering practices by automating critical tasks, including code generation, debugging and program repair. Despite these advancements, existing Automated Program Repair (APR) systems frequently suffer from complexity, high computational demands, and poor integration within practical software development lifecycles, particularly Continuous Integration and Continuous Deployment (CI/CD) workflows. Such shortcomings often lead to frequent context switching, which negatively impacts developer productivity. \break
In this thesis, we address these challenges by introducing a novel and lightweight APR system leveraging LLMs, explicitly designed for seamless integration into CI/CD pipelines deployed in budget constrained environments. Our containerized approach, developed with a strong emphasis on security and isolation, manages the complete bug-fixing lifecycle from issue creation on GitHub to the generation and validation of pull requests. By automating these processes end-to-end, the system significantly reduces manual intervention, streamlining developer workflows and enhancing overall productivity.\break
We evaluate our APR system using the QuixBugs benchmark, a recognized dataset for testing APR methodologies. The experimental results indicate that our streamlined and cost-effective solution effectively repairs small-scale software bugs, demonstrating practical applicability within typical software development environments. \break
The outcomes underscore the feasibility and advantages of integrating APR directly into real-world CI/CD pipelines. We also discuss limitations inherent in LLM-based solutions, such as accuracy and reliability issues and suggest future enhancement and research.
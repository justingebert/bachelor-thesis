\thispagestyle{empty}% keine Seitennummer

\section*{Abstract}
\ac{GenAI} is reshaping software engineering practices by enhancing and automating core tasks, including code generation, debugging and program repair. Despite these advancements, existing \ac{APR} approaches suffer from complexity, high computational demands, and a lack of integration within practical software development lifecycles. Such shortcomings often lead to frequent context switching, which negatively impacts developer productivity.

In this thesis, we address these challenges by introducing a novel and lightweight Automated Bug Fixing system leveraging \acp{LLM}, explicitly designed for seamless integration into \ac{CI} pipelines. Our containerized solution automates the bug-fixing lifecycle, from GitHub issue creation to the generation and validation of pull requests, reducing manual interventions and streamlining development workflows.

We evaluated our approach using the QuixBugs benchmark, testing twelve LLMs for effectiveness, efficiency, and costs. The results demonstrate repair success rates of up to 100\% with short execution times and low cost, highlighting the practicality of our streamlined solution in effectively repairing small-scale software bugs. 

The outcomes underscore the feasibility and potentials of integrating APR directly into \ac{CI} pipelines. Nevertheless, limitations such as reliability concerns, the non-deterministic nature of generated fixes, and potential security vulnerabilities are discussed. Enhancements like adaptive model selection and improved integration techniques, are outlined to further refine and optimize this promising approach in the future.

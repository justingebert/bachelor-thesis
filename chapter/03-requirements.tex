\section{Functional Requirements}

\begin{table}[ht]
    \centering
    \small
    \begin{tabular*}{\textwidth}{@{\extracolsep{\fill}} p{1cm} p{3cm} p{5cm} p{4cm} @{}}
        \toprule
        \textbf{ID} & \textbf{Title} & \textbf{Description} & \textbf{Verification} \\
        \midrule
        F0 & Issue Gathering
        & Query GitHub for all open issues labeled \texttt{BUG} in the repository.
        & \texttt{runAgent} logs list of fetched issue numbers and URLs. \\[4pt]
        F1 & Fetch Code
        & Fetch each QuixBugs problem into a fresh workspace (via \texttt{git clone} or local copy).
        & After F1, workspace/issue/ contains the correct source files. \\[4pt]
        F2 & Apply Patch
        & Apply an LLM-generated diff patch to the workspace files.
        & After patch, \texttt{git diff} output matches the patch payload. \\[4pt]
        F3 & Build \& Test
        & Run the project's test suite inside a Docker container and capture pass/fail status.
        & Docker exit code = 0 for pass and a JSON report file exists. \\[4pt]
        F4 & Report Results
        & Open a PR or post a comment on GitHub with the diff and summary metrics.
        & A PR or comment appears for each issue, showing diff and summary. \\[4pt]
        F5 & Modular Stages
        & Implement stages IssueGather, Fetch, Patch, Test, Report as separate modules.
        & Each stage can be toggled on or off via configuration without breaking the pipeline. \\[4pt]
        F6 & Metrics Collection
        & Log fix-rate, CI-cycle time, and sandbox events in CSV or JSON format.
        & A metrics file contains fields: issue, pass/fail, time, incidents. \\[4pt]
        F7 & Per-Issue Trigger
        & Execute F1--F4 for each fetched issue in sequence and aggregate results.
        & For N issues, produce N PRs (or "no-fix" comments) and N metric entries. \\[4pt]
        F8 & GitHub Integration
        & Use the GitHub API (with \texttt{GITHUB\_TOKEN}) to list issues, create branches, open PRs, and post comments.
        & All GitHub API calls succeed without manual intervention. \\
        \bottomrule
    \end{tabular*}
    \caption{Functional requirements (F0--F8)}
\end{table}

\section{Non-Functional and Safety Requirements}

\begin{table}[ht]
    \centering
    \small
    \begin{tabular*}{\textwidth}{@{\extracolsep{\fill}} p{1cm} p{3cm} p{5cm} p{4cm} @{}}
        \toprule
        \textbf{ID} & \textbf{Title} & \textbf{Description} & \textbf{Verification} \\
        \midrule
        N1 & Performance Budget
        & Total wall-clock time per issue run <= X minutes (including Docker startup).
        & Average CI-cycle time across issues <= X minutes. \\[4pt]
        N2 & Resource Limits
        & Docker container limited to <= X GB RAM and <= X CPU cores.
        & Launched with \texttt{--memory=Xg --cpus=X}; verify via container stats. \\[4pt]
        N3 & Reproducibility
        & Runs are deterministic given identical repo state and config.
        & Multiple runs on the same issue yield identical metrics. \\[4pt]
        N4 & Configurability
        & User can specify issue labels, timeouts, resource caps, and stage toggles via YAML or JSON.
        & Changing the config file alters agent behavior accordingly. \\[4pt]
        N5 & Scheduling Config
        & Workflow can be scheduled via cron or manually triggered (GitHub Actions workflow\_dispatch).
        & Adjusting the schedule in Actions YAML takes effect. \\[4pt]
        \midrule
        S1 & Filesystem Isolation
        & Prevent reads/writes outside the workspace directory (no escapes).
        & Attempts to access paths outside workspace/ are blocked and logged. \\[4pt]
        S2 & Network Whitelist
        & Block all outbound network traffic except to configured LLM API endpoints.
        & Non-LLM outbound connections are refused by Docker network policy. \\[4pt]
        S3 & Rollback on Failure
        & On test or policy failure, auto-reclone a fresh workspace copy before retry.
        & After failure, workspace resets to its pre-run state. \\[4pt]
        S4 & Command Whitelisting
        & Only allow predefined shell commands (e.g., git, pytest, npm test); block others.
        & Forbidden commands (e.g., rm -rf /) are denied. \\
        \bottomrule
    \end{tabular*}
    \caption{Non-Functional (N1--N5) and Safety (S1--S4) requirements}
\end{table}

\section{Benchmark Setup}

\begin{itemize}
    \item QuixBugs problems (Python).
    \item Prepare a base Docker image (e.g., \texttt{python:3.10}) with pytest and JSON-report support.
    \item Ensure each workspace clone is clean (no leftover artifacts).
    \item Record benchmark IDs and Docker image tags in a configuration file for reproducibility.
\end{itemize}

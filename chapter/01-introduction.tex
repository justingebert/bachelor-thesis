Generative AI is rapidly changing the software industry and how software is developed and maintained. The emergence of Large Language Models (LLMs), a subfield of Generative AI, has opened up new opportunities for enhancing and automating various domains of the software development lifecycle. Due to remarkable capabilities in understanding and generating code snippets, LLMs have become valuable tools for developers' everyday tasks such as requirement engineering, code generation, refactoring, and program repair \cite{houLargeLanguageModels2024, puvvadiCodingAgentsComprehensive2025}.
\\
Despite these advances, bug fixing remains a resource-intensive task and is often perceived negatively \cite{winterHowDevelopersReally2023}. It leads to frequent interruptions and context switching, which reduce developer productivity  \cite{vasilescuSkyNotLimit2016}
Bugs have direct impact on software quality by causing crashes, vulnerabilities or even data loss. \cite{tihanyiNewEraSoftware2024}
The process of bug fixing can be time-consuming and error-prone, leading to delays in software delivery and increased costs . %\cite{}
In fact, according to  CISQ: in 2022 alone poor software quality costs 2.41 trillion dollars only the US with at least 607 billion dollars spend on finding and fixing bugs \cite{CostPoorSoftware}.
\\
Given the critical role of debugging and bug fixing in software development, Automated Program Repair (APR) has gained significant research interest.
Typically, bug fixing involves multiple steps: bug reporting, localization, repair, and validation \cite{zhangEmpiricalStudyFactors2012, leeUnifiedDebuggingApproach2024,xiaAgentlessDemystifyingLLMbased2024,zhangPATCHEmpoweringLarge2025, wangEmpiricalResearchUtilizing2025}.
Recent research has shown that LLMs can effectively be used to enhance automated bug fixing, thereby introducing new standards in the APR world showing potential of making significant improvements in efficiency of the software development process \cite{xiaAgentlessDemystifyingLLMbased2024,liuMarsCodeAgentAInative2024,yangSWEagentAgentComputerInterfaces2024, sobaniaAnalysisAutomaticBug2023, xiaAutomatedProgramRepair2024, huCanGPTO1Kill2024}.
\\
However existing APR approaches are often complex and require significant computational resources \cite{rondonEvaluatingAgentbasedProgram2025}, making them less suitable for budget-constrained environments or individual developers. Additionally, the lack of integration with existing software development lifecycles and workflows limits their practical applicability in real-world development environments \cite{chenUnveilingPitfallsUnderstanding2025,liuMarsCodeAgentAInative2024}.
\\
Motivated by these challenges, this thesis explores the potential of integrating LLM based automated bug fixing within continuous integration and continuous deployment (CI/CD) pipelines. By leveraging the capabilities of LLMs, we aim to develop a cost-effective prototype for automated bug fixing that seamlessly integrates into existing software development workflows. Considering computational demands, complexity of integration and practical constraints we aim to provide insights into possibilities and limitations of out approach.



In conclusion, this thesis explored the integration of LLM-based automated bug fixing within \acf{CI} with a focus on implementation and evaluation of potentials and limitations of the approach. The developed prototype successfully demonstrated a seamless integration of an automated bug fixing solution into a GitHub repository, using GitHub Actions and containerization. This integration automates the process from issue creation to pull request generation, showcasing the potential of LLMs in modern software development workflows.

The evaluation on the QuixBugs benchmark showed promising results, with repair success rates of up to 100\% by using iterative prompting techniques leveraging execution environment information. Testing twelve different popular \acp{LLM} revealed that smaller, more cost-effective models can achieve good performance for small bugs. These models balance high repair success rates with rapid execution times and minimal costs, thus proving suitable for frequent integration into routine software development practices.

However, we also identified challenges and limitations of the integration, including the reliance on external LLM provider APIs, potential security and privacy vulnerabilities and restricted generalizability given the limited scope of the QuixBugs benchmark. Additionally, technical constraints of the GitHub Action CI platform can lead to workflow management issues, such as cluttered run histories.

Ultimately, this thesis contributes a practical, effective, and scalable framework for incorporating automated bug fixing into real-world software development environments, laying a foundation for future innovations and broader adoption of AI-driven \ac{APR} methodologies.